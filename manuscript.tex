\documentclass[a4paper,oneside,onecolumn,12pt]{LegrandOrangeBook}
\usepackage[utf8]{inputenc}
\usepackage[T1]{fontenc}
\def\magyarOptions{chapterhead=unchanged}
\usepackage[magyar]{babel}
\usepackage{longtable}
%\usepackage{amsthm}
% \usepackage{amsmath}
% \usepackage{amsfonts}
% \usepackage{txfonts}
% \usepackage{pxfonts}
% %\usepackage{eufrak}
% \usepackage{mathptmx}
\usepackage{setspace}
% \usepackage{graphicx}
\usepackage{parskip} %paragraph között legyen spacing
% \usepackage[export]{adjustbox}%kép pozicionálás
%\usepackage{minted}
\usepackage{svg}
% \usepackage{hyperref}
% \usepackage{multimedia}
% \usepackage{wrapfig}
% \usepackage{csquotes}
\usepackage{ragged2e}

% electronic ISBN
%\usepackage[SC5b,ISBN=000-00-000-0000-0]{ean13isbn}

% printed ISBN
\usepackage[SC5b,ISBN=000-00-000-0000-0]{ean13isbn}


\newcommand{\blankpage}{%
    \newpage
    \thispagestyle{empty}
    \mbox{}
    \newpage
}

\newcommand{\comment}[1]{{\textcolor{red}{#1}}}
\newcommand{\commentaron}[1]{{\textcolor{blue}{#1}}}

% Book information for PDF metadata, remove/comment this block if not required 
\hypersetup{
	pdftitle={A szakdolgozat címe}, % Title field
	pdfauthor={Szerző Neve}, % Author field
	pdfsubject={Szakdolgozat v. Diplomamunka}, % Subject field
	pdfkeywords={key1, key2, key3, key4}, % Keywords
	pdfcreator={LaTeX}, % Content creator field
}

\definecolor{ocre}{RGB}{243, 102, 25} % Define the color used for highlighting throughout the book

\chapterspaceabove{6.5cm} % Default whitespace from the top of the page to the chapter title on chapter pages
\chapterspacebelow{6.75cm} % Default amount of vertical whitespace from the top margin to the start of the text on chapter pages

\onehalfspacing
%\doublespacing
\frenchspacing

\hypersetup{
    colorlinks=true,
    linkcolor=blue,
    filecolor=magenta,      
    urlcolor=cyan,
    pdftitle={A szakdolgozat címe} }
\urlstyle{same}

%% \usepackage[backend=biber, style=numeric-comp, sorting=none,]{biblatex}
%% \PassOptionsToPackage{sorting=none,language=english}{biblatex}
\ExecuteBibliographyOptions{sorting=none}
\addbibresource{manuscript.bib}
\begin{document}

%%% Borító
\thispagestyle{empty}
\begin{minipage}[c][\textheight][c]{\textwidth}
	{\centering
	\includegraphics[keepaspectratio,width=3cm]{SelyeBanner.png}\\
	\vskip0.5cm
	{\LARGE UNIVERZITA J. SELYEHO}\\
	\vskip0.5cm
	{\LARGE SELYE JÁNOS EGYETEM}\\
    \vskip0.5cm
	{\large Fakulta ekonómie a informatiky}\\
	\vskip0.5cm
	{\large Gazdaságtudományi és Informatikai Kar}\\
	\vfill
	{\Huge A szakdolgozat címe}\\
	%\vskip2cm
	%{\Huge A ZÁRÓDOLGOZAT CÍME}\\
	%\vfill
	Szakdolgozat v. Diplomamunka\\
	Szerző Neve \\
    \ISBN\\
	\hfill\the\year{}, Komárom\hfill
	}
\end{minipage}

\cleardoublepage
\begingroup
\makeatletter
\let\ps@plain\ps@empty
\begin{minipage}[c][\textheight][c]{\textwidth}
	{\centering
	{\large UNIVERZITA J. SELYEHO\\SELYE JÁNOS EGYETEM}\\
	\vskip0.5cm
	{\ NÁZOV FAKULTY\\Fakulta ekonómie a informatiky\\Gazdaságtudományi és Informatikai Kar}\\
	\vfill
	{\Large NÁZOV PRÁCE\\Vytvorenie softvérového interpretera pre CHIP-8 }\\
	\vfill
	\thispagestyle{empty}
	\begin{tabular}{ll}
		Študijný program:    & Aplikovaná informatika \\
		Tanulmányi program:  & Alkalmazott Informatika\\
		Študijný odbor:      & Informatika\\
		Tanulmányi szak:     & Informatika\\
		Školiteľ:            & Tanár Neve\\
		Témavezető:          & Tanár Neve\\
		Konzultant:          & Tanár Neve \\
		Konzulens:           & Tanár Neve\\
		Školiace pracovisko: & Katedra informatiky\\
		Tanszék megnevezése: & Informatikai Tanszék\\
	\end{tabular}
	\vfill
	Označenie typu práce - Szakdolgozat v. Diplomamunka\\
	Szerző Neve\\
    \ISBN \\
	\hfill\the\year{}, Komárno\hfill
	}
\end{minipage}
\endgroup
{
\hspace*{-2cm}
%\includegraphics[keepaspectratio, width=17cm]{./zadanie-zp_17422.pdf}
Ide jön az aláírt témakiírás
}
\tableofcontents %% opcionális
\pagebreak
\listoffigures  %% opcionális
\addcontentsline{toc}{section}{Ábrák jegyzéke}

\pagebreak

\newcommand{\chpt}[1]{\chapter*{#1}\addcontentsline{toc}{section}{#1}}
\chapterimage{kep/header.png} % Chapter heading image

\chpt{Feladatkiírás}
% \addcontentsline{toc}{chapter}{Feladatkiírás}

A szerző egy feladatot oldott meg. Még egy kicsit hosszabb, még egy kicsit hosszabb, még egy kicsit hosszabb, még egy kicsit hosszabb, még egy kicsit hosszabb, még egy kicsit hosszabb, még egy kicsit hosszabb, még egy kicsit hosszabb, még egy kicsit hosszabb, még egy kicsit hosszabb, még egy kicsit hosszabb, még egy kicsit hosszabb, még egy kicsit hosszabb, még egy kicsit hosszabb, még egy kicsit hosszabb, még egy kicsit hosszabb, még egy kicsit hosszabb, még egy kicsit hosszabb, még egy kicsit hosszabb, még egy kicsit hosszabb, még egy kicsit hosszabb, még egy kicsit hosszabb, még egy kicsit hosszabb, még egy kicsit hosszabb, még egy kicsit hosszabb, még egy kicsit hosszabb, még egy kicsit hosszabb, még egy kicsit hosszabb, még egy kicsit hosszabb, még egy kicsit hosszabb, még egy kicsit hosszabb, még egy kicsit hosszabb, még egy kicsit hosszabb, még egy kicsit hosszabb, még egy kicsit hosszabb, még egy kicsit hosszabb, még egy kicsit hosszabb, még egy kicsit hosszabb, még egy kicsit hosszabb, még egy kicsit hosszabb, még egy kicsit hosszabb, még egy kicsit hosszabb, még egy kicsit hosszabb, még egy kicsit hosszabb, még egy kicsit hosszabb, még egy kicsit hosszabb, még egy kicsit hosszabb,  

\chapterimage{kep/header2.png} % Chapter heading image
\chpt{Opis práce}
% \addcontentsline{toc}{chapter}{Opis práce}

Autor vyrešil úlohu, ešte trošku dlhšie, ešte trošku dlhšie, ešte trošku dlhšie, ešte trošku dlhšie, ešte trošku dlhšie, ešte trošku dlhšie, ešte trošku dlhšie, ešte trošku dlhšie, ešte trošku dlhšie, ešte trošku dlhšie, ešte trošku dlhšie, ešte trošku dlhšie, ešte trošku dlhšie, ešte trošku dlhšie, ešte trošku dlhšie, ešte trošku dlhšie, ešte trošku dlhšie, ešte trošku dlhšie, ešte trošku dlhšie, ešte trošku dlhšie, ešte trošku dlhšie, ešte trošku dlhšie, ešte trošku dlhšie, ešte trošku dlhšie, ešte trošku dlhšie, ešte trošku dlhšie, ešte trošku dlhšie, ešte trošku dlhšie, ešte trošku dlhšie, ešte trošku dlhšie, ešte trošku dlhšie, ešte trošku dlhšie, ešte trošku dlhšie, ešte trošku dlhšie, ešte trošku dlhšie, ešte trošku dlhšie, ešte trošku dlhšie, ešte trošku dlhšie, ešte trošku dlhšie, ešte trošku dlhšie, ešte trošku dlhšie, ešte trošku dlhšie, ešte trošku dlhšie, ešte trošku dlhšie
\pagebreak

\chapterimage{kep/header3.png} % Chapter heading image
\chpt{Abstrakt}\label{sec:abstrakt}
% \addcontentsline{toc}{chapter}{Abstrakt}

ešte trošku dlhšie, ešte trošku dlhšie, ešte trošku dlhšie, ešte trošku dlhšie, ešte trošku dlhšie, ešte trošku dlhšie, ešte trošku dlhšie, ešte trošku dlhšie, ešte trošku dlhšie, ešte trošku dlhšie, ešte trošku dlhšie, ešte trošku dlhšie, ešte trošku dlhšie, ešte trošku dlhšie, ešte trošku dlhšie, ešte trošku dlhšie, ešte trošku dlhšie, ešte trošku dlhšie, ešte trošku dlhšie, ešte trošku dlhšie, ešte trošku dlhšie, ešte trošku dlhšie, ešte trošku dlhšie, ešte trošku dlhšie, ešte trošku dlhšie, ešte trošku dlhšie, ešte trošku dlhšie, ešte trošku dlhšie, ešte trošku dlhšie, ešte trošku dlhšie, ešte trošku dlhšie, ešte trošku dlhšie, ešte trošku dlhšie, ešte trošku dlhšie, ešte trošku dlhšie, ešte trošku dlhšie, ešte trošku dlhšie, ešte trošku dlhšie, ešte trošku dlhšie, ešte trošku dlhšie, ešte trošku dlhšie, ešte trošku dlhšie, ešte trošku dlhšie, ešte trošku dlhšie

\textbf{Kľúčové slová: klúč1, klúč2, klúč3, }
\pagebreak

\chapterimage{kep/header4.png} % Chapter heading image
\chpt{Absztrakt}\label{sec:absztrakt}
% \addcontentsline{toc}{chapter}{Absztrakt}

A szerző egy feladatot oldott meg. Még egy kicsit hosszabb, még egy kicsit hosszabb, még egy kicsit hosszabb, még egy kicsit hosszabb, még egy kicsit hosszabb, még egy kicsit hosszabb, még egy kicsit hosszabb, még egy kicsit hosszabb, még egy kicsit hosszabb, még egy kicsit hosszabb, még egy kicsit hosszabb, még egy kicsit hosszabb, még egy kicsit hosszabb, még egy kicsit hosszabb, még egy kicsit hosszabb, még egy kicsit hosszabb, még egy kicsit hosszabb, még egy kicsit hosszabb, még egy kicsit hosszabb, még egy kicsit hosszabb, még egy kicsit hosszabb, még egy kicsit hosszabb, még egy kicsit hosszabb, még egy kicsit hosszabb, még egy kicsit hosszabb, még egy kicsit hosszabb, még egy kicsit hosszabb, még egy kicsit hosszabb, még egy kicsit hosszabb, még egy kicsit hosszabb, még egy kicsit hosszabb, még egy kicsit hosszabb, még egy kicsit hosszabb, még egy kicsit hosszabb, még egy kicsit hosszabb, még egy kicsit hosszabb, még egy kicsit hosszabb, még egy kicsit hosszabb, még egy kicsit hosszabb, még egy kicsit hosszabb, még egy kicsit hosszabb, még egy kicsit hosszabb, még egy kicsit hosszabb, még egy kicsit hosszabb, még egy kicsit hosszabb, még egy kicsit hosszabb, még egy kicsit hosszabb, 

\textbf{Kulcsszavak: kulcs1, kulcs2, kulcs3}

\pagebreak

\chapterimage{kep/header.png} % Chapter heading image
\chpt{Abstract}
Little longer, Little longer, Little longer, Little longer, Little longer, Little longer, Little longer, Little longer, Little longer, Little longer, Little longer, Little longer, Little longer, Little longer, Little longer, Little longer, Little longer, Little longer, Little longer, Little longer, Little longer, Little longer, Little longer, Little longer, Little longer, Little longer, Little longer, Little longer, Little longer, Little longer, Little longer, Little longer, Little longer, Little longer, Little longer, Little longer, Little longer, Little longer, Little longer, Little longer, Little longer, Little longer, Little longer, Little longer, Little longer, Little longer, Little longer, Little longer, Little longer, Little longer, Little longer, Little longer, Little longer, Little longer, Little longer, Little longer, Little longer, 

\textbf{Keywords: key1, key2, key3}

\pagebreak

\chapter*{Bevezetés}
\addcontentsline{toc}{chapter}{Bevezetés}
\markboth{}{\sffamily\normalsize{Bevezetés}}
\begin{eBox}
    Fontos megjegyzés, vagy állítás
\end{eBox}
% \commentaron{szürke háttérrel kicsit fura nekem}\comment{válassz tetszőleges hátteret}

\begin{figure}
  \begin{tBox}
    \centering{\LARGE{\url{https://google.com/}}}
  \end{tBox}
  \caption{Felirat}\label{fig:link}
\end{figure}  

Még egy kicsit hosszabb, még egy kicsit hosszabb, még egy kicsit hosszabb, még egy kicsit hosszabb, még egy kicsit hosszabb, még egy kicsit hosszabb, még egy kicsit hosszabb, még egy kicsit hosszabb, még egy kicsit hosszabb, még egy kicsit hosszabb, még egy kicsit hosszabb, még egy kicsit hosszabb, még egy kicsit hosszabb, még egy kicsit hosszabb, még egy kicsit hosszabb, még egy kicsit hosszabb, még egy kicsit hosszabb, még egy kicsit hosszabb, még egy kicsit hosszabb, még egy kicsit hosszabb, még egy kicsit hosszabb, még egy kicsit hosszabb, még egy kicsit hosszabb, még egy kicsit hosszabb, még egy kicsit hosszabb, még egy kicsit hosszabb, még egy kicsit hosszabb, még egy kicsit hosszabb, még egy kicsit hosszabb, még egy kicsit hosszabb, még egy kicsit hosszabb, még egy kicsit hosszabb, még egy kicsit hosszabb, még egy kicsit hosszabb, még egy kicsit hosszabb, még egy kicsit hosszabb, még egy kicsit hosszabb, még egy kicsit hosszabb, még egy kicsit hosszabb, még egy kicsit hosszabb, még egy kicsit hosszabb, még egy kicsit hosszabb, még egy kicsit hossz

\chapterimage{kep/header2.png} % Chapter heading image
\chapter{Elméleti rész}
Öntözőrendszerek
Élelmezési problémák
	Jelen pillanatban a Föld lakosságának pontos számát meghatározni rendkívüli kihívást jelent, mivel ez egy olyan változó, mely dinamikusan változtatja az értékét. A legpontosabb adatok megszerzése az országos népszámlálások és a születési valamint halálozási arányok figyelembevételével történik. Ezen módszerek és adatok legpontosabb figyelembevételével is fennáll egy 1-2% hibahatár, ami egy ilyen hatalmas populáció esetén jelentős eltéréseket eredményezhet. Ezen adathiány és hibahatár miatt a világ lakosságának pontos számának meghatározása egy felettébb nehéz probléma. Viszont jó megközelítéssel egy elfogadható értéket vagyunk képesek megkapni, amely rálátást biztosít a föld lakosságának számára
	Az ENSZ adatai alapján, 2022 november 15-én, a világ lakosságának száma elérte a 8 milliárd főt, és ezen a számon továbbra is növekedés tapasztalható. Ez nem elhanyagolható demográfiai problémákat vet fel, legfőképpen az élelmiszerellátás terén. 
	Mivel Földünk megművelhető és élelmiszer élőállítására megfelelő földterület egy olyan paraméter amely fixen adott, így az elkövetkezendő években a lakosság növekedése további nyomást helyez az élelmiszertermelés és az ellátási láncokra. Az élelmiszeripar és mezőgazdaság kénytelen lesz nagyobb mennyiségű élelmiszert termelni, hogy az embereket élelemmel lássa el. Emellett meg kell oldania az élelmiszer elosztás és fenntarthatóság kérdését is. 
	A legfőbb kitermelési illetve ellátási problémák olyan térségekben tapasztalhatók amelyeknek térségi adottságai megnehezítik vagy esetekben teljesen ellehetetlenítik a mező- illetve egyéb más gazdaságok létjogosultságát. Ilyen övezetek például Észak-Afrika valamint a Földközi-tenger térségei. Ezen területek nagy mértékben adományok valamint segélyszervezetek munkájától  függnek, viszont ezen cselekedetek nem mozdítják előre, hanem ellenkező esetben lecsökkentik az önállósodási hajlamot. Így ezekben az országokban ahol nincs ráfordított energia és erőforrás olyan létesítmények létrehozására amely az élelmezési lehetőségek problémait szándékoznának megoldani jelentősen megnövekedhet a kivándorlási arányok száma. Így az élelmezés mint paraméter nagy mértékben képes befolyásolni azt. Ezen területet országaiban kicsit sem  elhanyagolható, mivel a lakosság egyre nő, viszont az éghajlati adottságok nagyban megnehezítik a termelést. Egy tunéziai kutatócsapat kimutatásai alapján (2017) rendkívül fontos a mezőgazdaság termelés fenntartása és annak jelentős újraszervezése, mivel az nemcsak élelmiszert képes kitermelni, hanem jelen esetünkben munkahelyeket is képes biztosítani, így csökkentve a kivonulás mértékét.  
	Globális mértékben a 2022-es élelmezési biztonsági jelentés szerint, 2021-ben az éhezők száma elérte a 828 millió főt. Az ENSZ-jelentése a globális válság fő okait a mélyszegénységben és az országok közötti, valamint az országokon belüli növekvő egyenlőtlenségekben látja. Emellett hangsúlyozza, hogy a Covid-19 járvány, a háborús konfliktusok, szélsőséges időjárásiesemények és ezek gazdasági hatásai is közrejátszanak a válság kialakulásában. A jelentés kiemeli ezen tényezőknek kölcsönhatásait, amikor a probléma gyökerét elemzi. 
Ezen túlmenően, a nem fenntartható mezőgazdaság és élelmiszertermelés következményeire és az élelmezési rendszerek torzulásaira is rámutat. Az Ukrajna elleni orosz agresszió tovább rontja a már drámai helyzetet. Az ukrajnai gabonaszállítások megszűnése nem csak az érintett államokra van hatással, hanem különösen az élelmiszerimportra szoruló afrikai és közel-keleti országokra, valamint néhány ázsiai országra, akik eddig jelentős mennyiségű gabonát szereztek be a térségből, és most az élelmiszer-ellátásuk kerül veszélybe.

A mezőgazdaság főbb alkotóelemei
A mezőgazdasági termelését számos tényező befolyásolja, mint a természeti környezet és a gazdasági tényezők. Ezek összetett módon kapcsolódnak egymáshoz. A természeti tényezők, például az éghajlat, víz és talajminőség, fontosak a mezőgazdaságban. A természeti-ökológiai viszonyok határozzák meg, milyen növények és állatok terjeszthetők ott, de a fejlesztések segíthetnek a termelés területének növelésében. Ugyanabban a régióban néha ugyanolyan természeti feltételek mellett többféle növényt termesztenek, de eltérő hozamokat érnek el.
Táptalaj
A fent említett demográfiai és környezeti problémák komoly kihívást jelentenek a természeti erőforrásokra, különösen a termőföldre, amely az agrárgazdaság egyik legfontosabb alapja. A termőföld mennyiségének csökkenése olyan széles körű következményekkel járhat, mint az állattenyésztés jelentős visszaesése. Ennek eredményeként az állatállomány mérséklődik, és ezáltal az állati termékek ellátása is csökken, ami közvetve hat az emberi táplálkozásra. Az élelmiszertermelés területén is hasonlóan komoly problémákat okozhat a termőföldek csökkenése. A zöldség- és gyümölcstermelésben is hatalmas visszaesést okozhat.
A talaj fizikai tulajdonságai, például porozitás, sűrűség, vízfelvételi és légáteresztő képesség, függnek a talaj mechanikai összetételétől, beleértve a szemcsék méretét és az agyag-homok arányát. Ezek a tulajdonságok meghatározzák, mennyire alkalmas egy talaj a mezőgazdasági termelésre. Például, talajok, amelyekben sok finom agyag és iszapszemcse található, kötöttek és nehezen művelhetők. Bár jól tartják vissza a nedvességet, vízvezető képességük gyenge. Másrészt, talajok, amelyekben sok homok van, könnyen vízfelvételre képesek, de nehezen tartják meg a vizet. Emellett a homoktalajok általában szegények humuszban az agyagos talajokhoz képest.
Az ember évszázadokon keresztül csak azt tudta betakarítani a mezőgazdaságból, amit a termőföld önmagában képes volt termelni. Hiányzó tápanyagok esetén csak szerves trágyákat használtak a pótlásra. Az ipar fejlődésével ma már ipari eljárásokkal elő lehet állítani az alapvető tápanyagokat, mint a nitrogén, foszfor, kálium, és mész, és ezeket a mérsékelten termékeny talajokba lehet juttatni a termények növelése érdekében.
A termőföld számos tényező függvénye, ami kiterjedt földrajzi, környezeti és gazdasági tényezőkből tevődik össze. Az egyes régiókban jelentős eltérések mutatkoznak a megművelt területek és a mezőgazdaságilag hasznosítható termőföldek eloszlásában. Például Európában a megművelt területek aránya a legmagasabb (88%), míg Dél-Amerikában a legalacsonyabb (11%).
Világviszonylatban a művelhető területek kevesebb, mint felét művelik. Ennek eredményeként a mezőgazdasági termelés sok helyen nem tudja elérni a kívánt mennyiséget és minőséget, különösen azokon a területeken, ahol a növekvő népesség miatt jelentős élelmiszerhiány alakul ki.
Éghajlat
Az éghajlati viszonyoknak kiemelten fontos szerepük van a mezőgazdasági termelés területi elhelyezkedésének meghatározásában. Az éghajlati tényezők, mint például hőmérséklet, napsütéses órák, légnyomás és csapadék, együtt határozzák meg, hogy milyen növényeket és állatokat lehet egy adott területen gazdaságosan termeszteni. 
Az éghajlati viszonyok mint például a hőmérséklet és a csapadék, képesek meghatározni, hogy melyik kultúrnövény vagy állatfaj mely területeken képes a legjobban a gyarapodásra Például a gabonafélék és a burgonya, amelyek széles hőmérsékleti tartományban nőnek, elterjedhetnek a sztyeppeterületektől egészen a tundráig. Azonban olyan növények, mint a rizs, specifikus környezeti feltételeket igényelnek, például jelentős mennyiségű vízre van szükségük a termesztéshez. Ha a víz rendelkezésre áll, a rizs elterjedhet trópusi övezettől a mérsékelt övezetig (így akár Magyarországon is). 
Az éghajlati tényezők közül a hőmérséklet kiemelten fontosságú a mezőgazdasági termelés szempontjából. A hőmérséklet meghatározza, hogy mely növényi kultúrák termeszthetőek bizonyos területeken. Minden növénynek megvan a maga saját hőigénye, és vannak optimális hőmérsékleti tartományok amelyekben a növények a legjobban fejlődnek. Túl hideg vagy túl meleg időjárás negatívan befolyásolja a terméshozamot és bizonyos esetekben akár a termelési költségeket is növelheti
Minél jobban eltérünk egy adott növény hőmérsékleti optimumától, annál rosszabb lesz a terméshozam és annál magasabbak lesznek a termelési költségek.
A növények hőigénye eltérő, különösen növekedésük különböző szakaszaiban. Például a trópusi növények legalább 18-20°C éves átlaghőmérsékletet igényelnek, míg a mérsékelt éghajlaton télen az évelő növények nyugalmi állapotban vannak. Néhány növény, például alma és körte, hideg hőmérsékletre van szükség a megfelelő fejlődéshez. Egyetlen fagyos éjszaka például jelentősen csökkentheti a kukorica termésátlagát az Egyesült Államok Mississippi-medence északi részén.
Azonban ma már nemcsak a hideg okoz problémákat a mezőgazdaságban. A magas nyári hőmérséklet is jelentős károkat okozhat a mérsékelt övezetben. A magas hő és párolgás miatt sok növény vízhiányos állapotba kerülhet, és néhány területen teljesen el is pusztulhat. A hőmérsékleti adottságok időtartama is kulcsfontosságú. Például a búza számára legalább 4-5 hónapos 10°C feletti átlaghőmérsékletre van szükség a fejlődéshez és beéréshez. Ezért például a búzát Dániában és Svédország déli részén is termesztik, ahol ezek az átlaghőmérsékletek elérhetők. 
Mindezek mellet a légáramlások is fontos szerepet játszanak a mezőgazdasági termelésben, mivel hatással vannak a hőmérsékletre és a csapadékra. A szél közvetlen hatása mellet még a hőmérsékletre és a csapadék mennyiségére is hatással van. 
A trópusi övezetben gyakran előforduló pusztító ciklonokhoz, mint például a tornádók, hurrikánok és tájfunokhoz kell alkalmazkodnia a növénytermesztésnek. Ezért alacsony szárú növényzetet termesztenek a zárt, magas növényzet helyett.
	Észak-Afrikában és Ausztrália sivatagi területein időszakosan kialakuló forró, száraz porviharok károsítják az oáziskultúrákat és kiszárítják a természetes növényzetet és legelőket.
Az amerikai kontinensen az észak felől érkező hideg légtömegek olykor egészen a Mexikói-öbölig juthatnak, ami károkat okozhat a kukorica, dohány, gyapot, rizs és cukornádültetvények számára. A floridai banán, ananász, citrom, narancs és grapefruit ültetvények is fagykárokat szenvedhetnek el.
A lokális szelek is fontos szerepet játszanak a mezőgazdaságban. Például a főn, amely a magashegységekben jelentkezik, befolyásolja a csapadékot és hőmérsékletet. A magashegységek szélnek kitett oldalán bő csapadék keletkezik, míg az árnyékos oldalon alacsonyabb hőmérséklet lehetővé teszi a kukorica, szőlő, cseresznye és kajszibarack termesztését. Ugyanakkor a "hideg főn" például veszélyezteti a citrom, narancs, levendula, mandula és virágültetvényeket.

Víz valamint az öntözés
	Egy a természetben előforduló szintén életünk alappilléreként szolgáló erőforrásra is nagy nyomás és figyelem hárul. Ez nem más mint a víz, amely a termőföldhöz hasonlóan szintén korlátozott mennyiségű. 
	A mezőgazdasági termelés sikeréhez alapvetően szükséges, hogy elegendő víz álljon rendelkezésre. A víz kulcsszerepet játszik minden mezőgazdasági szektorban, beleértve a növénytermesztést, az állattenyésztést és az erdőgazdálkodást is. A Föld különböző területein változó természetes vízforrások vannak jelen, és ezek erősen összefüggenek az éghajlati körülményekkel. Ennek következtében az egyes régiók eltérő természetes vízforrásokkal rendelkeznek, amelyek meghatározzák a mezőgazdasági termelés lehetőségeit és korlátait ezeken a területeken.
	Abban az esetben, ha a rendelkezésre álló víz mennyisége nem elégíti ki a növénytermesztés vagy állattenyésztés vízigényét, akkor az negatívan negatívan befolyásolhatja a növények növekedését és az állatok súlygyarapodását. Ez aszályhoz vezethet, aminek eredményeként a növények és az állatok elpusztulhatnak. Fontos megjegyezni, hogy a növények és az állatok igyekeznek alkalmazkodni a rendelkezésre álló vízmennyiséghez. Azonban a természetes egyensúly, amely a vízellátottság és a vízigény között kialakul, néha megbomlik, és ekkor lehet szükség a vízigény mesterséges pótlására, akár részben, akár teljesen.
	A mezőgazdasági termelés szempontjából épp olyan veszélyt jelent a túlzott vízmennyiség mint a vízhiány. Eme tényező nagyban képes befolyásolni, különösen a növénytermesztés szempontjából. A talaj túlzott nedvessége miatt a levegő kiűződik a talajból, ami gátolja a növények növekedését és negatívan hat az életfolyamataikra. A talaj tartós vagy hosszantartó túlzott nedvessége károsan érinti a mezőgazdasági termelést. Gyakran ilyen körülmények között termésvesztésekkel kell számolni, hosszú távon pedig a talaj termőképességének jelentős csökkenésével, vagy akár pusztulásával.
	A világ különböző régióiban termesztett növényeknek eltérő vízigényük van. Az éghajlati változatosság és a származási hely miatt a növények vízigénye sokféle lehet. A növények génjeikben hordozzák ezt a vízigényüket, még akkor is, ha más éghajlati és vízellátási körülmények között termesztik őket. Például olyan területeken, ahol kevésbé kedvező a természetes vízellátás, alacsonyabb terméshozamra és növekvő termelési kockázatra lehet számítani a rosszabb vízellátottság miatt.
	A növénytermesztés fő célja a magas terméshozam elérése. Ennek érdekében olyan térségekben, ahol megbomlik a vízigény és a vízellátás egyensúlya, időnként szükségessé válhat a mesterséges vízpótlás. Ezért a növénytermesztés két lehetőséget kínál. Az egyik lehetőség az, hogy a termesztett növények alkalmazkodnak a rendelkezésre álló természetes vízmennyiséghez, míg a másik lehetőség az, hogy mesterségesen pótoljuk a vízhiányt a növények igényei szerint.
	Az öntözhető területek növekedése a 19. században indult meg dinamikusan. A század elején a világban összesen mintegy 8 millió hektárnyi terület volt öntözhető. A század végére ez a terület már 40 millió hektárra nőtt. A 20. században tovább folytatódott az öntözött területek növekedése, és napjainkra már meghaladja a 255 millió hektárt. Fontos megjegyezni, hogy az öntözhető területek növekedési üteme lassult, mivel a könnyen öntözhető területeken ezt a feladatot már sikerrel megoldották. A kevésbé kedvező természeti körülményekkel rendelkező területeket jóval nehezebb és költségesebb öntözni.
	Az öntözött területek térbeli eloszlását jelentősen befolyásolják az éghajlati tényezők és a természetes vízellátottság. Az éghajlati övezetek alapján az öntözött területeket a következő kategóriákra lehet osztani:
-	Azokon a területeken és országokban, ahol a száraz éghajlat vagy csak nagyon korlátozottan teszi lehetővé a növénytermesztést öntözés nélkül, az öntözés elengedhetetlen a mezőgazdasági termelés számára. Ilyen területek és országok közé tartoznak például Pakisztán, Üzbegisztán, Irak, Irán jelentős része, Izrael, Észak-Afrika arab országai és Egyiptom. Ezekben a területeken az öntözött területek nagyon magas részarányt képviselnek. Például Pakisztánban az öntözött területek aránya eléri a 80%-ot, Üzbegisztánban 89%-ot, míg Egyiptomban 100%-ot.
-	Azok a területek és országok, ahol az év jelentős részében lehetőség van növénytermesztésre öntözés nélkül, de a nagyobb hozamok elérése érdekében kiegészítő öntözésre van szükség. Az öntözésbe vont területek nagy része ebbe a kategóriába tartozik. Ilyen területek közé tartoznak például a trópusi éghajlatú területeken Mianmar és Vietnám szárazabb vidékei, Kínában Hszincsing tartomány, a thaiföldi Chao-Praya vidéke, Európa bizonyos részei, Ausztrália, az Amerikai Egyesült Államok Kalifornia mediterrán és egyes mérsékelt éghajlatú területei is. Az öntözés itt hozzájárul a termelés növeléséhez és a megbízhatóbb mezőgazdasági termeléshez.
-	Azok a területek és országok, ahol a természetes vízellátottság lehetővé teszi az évi legalább egyszeri biztonságos termés betakarítását, de az egyéb kedvező éghajlati tényezők (magas napsütéses idő, hőellátottság) teljes mértékű kihasználásához öntözésre van szükség. Ezekben a térségekben előfordulhat, hogy néhány területen a növényeknek kétszer vagy akár háromszor is lehetőség nyílik a termények betakarítására. Azonban ehhez szigorú és körültekintő földgazdálkodásra van szükség. Ide tartoznak azok a területek és országok, amelyekre főként a monszun éghajlat jellemző, mint például Kína délkeleti tartományai, Banglades, Fülöp-szigetek és Srí Lanka nedvesebb éghajlatú vidékei.

Az öntözőrendszerek kiépítése és az öntözési technológia fejlesztése napjainkban igen tőkeigényes folyamat. Ennek eredményeként általában csak olyan országok tudják saját forrásaikból biztosítani a szükséges tőkét az öntözőrendszerek kiépítéséhez, amelyek gazdagok nyersanyagokban, különösen olajban. Tőkehiányos országok esetében, például India, Pakisztán, Banglades és egyes afrikai országok esetében, a nagyobb méretű beruházásokhoz jelentős nemzetközi együttműködésre, például az ENSZ és az FAO segítségére van szükség.
A mezőgazdasági termelés, főként a növénytermesztés, nagy részben a természetes csapadékra támaszkodik a vízigény kielégítéséhez. 
Az éves csapadék mennyisége, intenzitása és halmazállapota számos tényezőtől függ. Ezek közé tartozik a földrajzi szélesség, a távolság a tengertől, a tengerszint feletti magasság, a felszíndomborzat, a lejtés iránya (luv, vagyis a csapadéknak kitett területek és lee, azaz a szélárnyékos területek; déli lejtők esetén pozitív hőmérsékleti hatás, míg északi lejtők esetén negatív hőmérsékleti hatás), a felszín anyagi és alaki tulajdonságai, valamint az uralkodó szélirány. 
A lehullott csapadék mennyisége önmagában még nem határozza meg a mezőgazdasági termelés feltételeit. Ennek hatása szorosan összefügg a hőmérséklettel. Magas hőmérséklet esetén a lehullott csapadék gyorsabban párolog, mint alacsony hőmérsékleten. Ennek eredményeként a magas hőmérséklet mellett lehullott csapadék kevésbé hatékonyan hasznosítható a mezőgazdaságban. Például Dél-Ukrajnában az éves 500 mm-es csapadékmennyiség alig elegendő a növények vízigényének kielégítéséhez, míg Nyugat-Szibériában ugyanaz a mennyiség már vízbőséget eredményez. Ennek oka, hogy a magasabb évi középhőmérséklet mellett általában több csapadék szükséges a mezőgazdasági termeléshez, míg alacsonyabb hőmérséklet mellett kevesebb. Ugyanakkor fontos megjegyezni, hogy az egyes növények vízigényét a termés mennyisége, valamint a hőmérséklet, a fényviszonyok és a talajminőség is befolyásolják.
A csapadék mennyisége jelentős hatással van a mezőgazdasági hozamokra. Területek, ahol több csapadék esik, általában kedvezőbb feltételeket kínálnak a mezőgazdasági termeléshez, mivel kisebb költségekkel nagyobb terméseket lehet elérni. A mezőgazdasági termelés szempontjából az éves csapadék mennyisége mellett az eloszlása is fontos tényező. Ennek oka, hogy a növények vízigénye a fenofázisok előrehaladtával változik és növekszik. A növények a legtöbb vizet az erőteljes növekedés és a virágzási időszak során igénylik leginkább, például a gabonafélék a bokrosodástól a kalászképződésig.
A mérsékelt övezetben általában a nyári hónapokban, a legmelegebb időszakban hull a legtöbb csapadék, míg a téli hónapokban kevesebb. A téli csapadék mennyisége tavasszal fontos a talajban tárolt vízkészlet szempontjából, mivel ez az időszak, amikor a növények aktív növekedésbe kezdenek. Ennek eredményeként a tavaszi és nyári kezdetű csapadékhiány kockázatos lehet, mivel a vegetáció ekkor a legintenzívebb.
Emellett a mérsékelt övezetben a nyári csapadék hiánya, vagyis az aszály is súlyos következményekkel járhat, például terméskiesést és alacsonyabb hozamokat eredményezhet. A mediterrán övezetben a rendszertelen és kevés nyári csapadék miatt vízhiány jelentkezhet, ami komoly mezőgazdasági kockázatot jelen. A mérsékelt övezetben a csapadék általában egyenletesen oszlik el, bár megfigyelhető egy markáns nyári csapadékmaximum. Trópusi övezetekben pedig az esőzések periódusokban érkeznek, ami két vagy egy évszakonként változhat.
A Föld megművelt területeinek jelentős részén a különböző növényi kultúrák vízszükséglete nem mindenhol kielégíthető. Mintegy a kontinensek területének negyedét teszik ki száraz, félsivatagi vagy sivatagi területek. Ezek a területek vízhiány miatt nem alkalmasak mezőgazdasági termelésre, vagy csak korlátozott mértékben, általában extenzív módon lehet ott mezőgazdaságot folytatni. 
A száraz éghajlatú területeken a párolgás mértéke állandóan vagy tartósan meghaladja a lehullott csapadék mennyiségét, ami azt eredményezi, hogy a vízháztartás mérlege negatív. A csapadékhiánnyal jellemezhető szárazsági határ jelentős ingadozásokat mutathat. A területeken, ahol kedvező talajadottságok és alacsony párolgás van, a csapadékvíz megfelelő visszatartása és a talajban történő raktározása lehetővé teszi a mezőgazdasági termelést akár 250-300 mm éves csapadékmennyiség mellett is. 
A Föld megművelt területeinek jelentős részén a különböző növényi kultúrák vízszükséglete nem mindenhol kielégíthető. Mintegy a kontinensek területének negyedét teszik ki száraz, félsivatagi vagy sivatagi területek. Ezek a területek vízhiány miatt nem alkalmasak mezőgazdasági termelésre, vagy csak korlátozott mértékben, általában extenzív módon lehet ott mezőgazdaságot folytatni. 
A száraz éghajlatú területeken a párolgás mértéke állandóan vagy tartósan meghaladja a lehullott csapadék mennyiségét, ami azt eredményezi, hogy a vízháztartás mérlege negatív. A csapadékhiánnyal jellemezhető szárazsági határ jelentős ingadozásokat mutathat. A területeken, ahol kedvező talajadottságok és alacsony párolgás van, a csapadékvíz megfelelő visszatartása és a talajban történő raktározása lehetővé teszi a mezőgazdasági termelést akár 250-300 mm éves csapadékmennyiség mellett is.
Ugyanakkor a trópusi övezet határterületein, ahol jelentős párolgás és hosszú száraz évszak van, a termeszthetőség akár 1000 mm éves csapadékmennyiség mellett is jelentős kockázatot jelenthet. Összességében a mezőgazdasági termelés legfőbb korlátja nem az éves csapadékmennyiség mennyisége, hanem annak hiánya.







\section{Alfejezet}

Még egy kicsit hosszabb, még egy kicsit hosszabb, még egy kicsit hosszabb, még egy kicsit hosszabb, még egy kicsit hosszabb, még egy kicsit hosszabb, még egy kicsit hosszabb, még egy kicsit hosszabb, még egy kicsit hosszabb, még egy kicsit hosszabb, még egy kicsit hosszabb, még egy kicsit hosszabb, még egy kicsit hosszabb, még egy kicsit hosszabb, még egy kicsit hosszabb, még egy kicsit hosszabb, még egy kicsit hosszabb, még egy kicsit hosszabb, még egy kicsit hosszabb, még egy kicsit hosszabb, még egy kicsit hosszabb, még egy kicsit hosszabb, még egy kicsit hosszabb, még egy kicsit hosszabb, még egy kicsit hosszabb, még egy kicsit hosszabb, még egy kicsit hosszabb, még egy kicsit hosszabb, még egy kicsit hosszabb, még egy kicsit hosszabb, még egy kicsit hosszabb, még egy kicsit hosszabb, még egy kicsit hosszabb, még egy kicsit hosszabb, még egy kicsit hosszabb, még egy kicsit hosszabb, még egy kicsit hosszabb, még egy kicsit hosszabb, még egy kicsit hosszabb, még egy kicsit hosszabb, még egy kicsit hosszabb, még egy kicsit hosszabb, még egy kicsit hossz

\section{Mégegy alfejezet}

felsorolás

\begin{itemize}
    \item LLE - {\it low-level} emuláció - alacsony szintű emulátorok %- ciklus pontosság
    \item HLE - {\it high-level} emuláció - magas szintű emulátorok %- absztrakciók
\end{itemize}


\subsection{Al-alfejezet}

Még egy kicsit hosszabb, még egy kicsit hosszabb, még egy kicsit hosszabb, még egy kicsit hosszabb, még egy kicsit hosszabb, még egy kicsit hosszabb, még egy kicsit hosszabb, még egy kicsit hosszabb, még egy kicsit hosszabb, még egy kicsit hosszabb, még egy kicsit hosszabb, még egy kicsit hosszabb, még egy kicsit hosszabb, még egy kicsit hosszabb, még egy kicsit hosszabb, még egy kicsit hosszabb, még egy kicsit hosszabb, még egy kicsit hosszabb, még egy kicsit hosszabb, még egy kicsit hosszabb, még egy kicsit hosszabb, még egy kicsit hosszabb, még egy kicsit hosszabb, még egy kicsit hosszabb, még egy kicsit hosszabb, még egy kicsit hosszabb, még egy kicsit hosszabb\footnote{lábjegyzet}, még egy kicsit hosszabb, még egy kicsit hosszabb, még egy kicsit hosszabb, még egy kicsit hosszabb, még egy kicsit hosszabb, még egy kicsit hosszabb, még egy kicsit hosszabb, még egy kicsit hosszabb, még egy kicsit hosszabb, még egy kicsit hosszabb, még egy kicsit hosszabb, még egy kicsit hosszabb, még egy kicsit hosszabb, még egy kicsit hosszabb, még egy kicsit hosszabb, még egy kicsit hossz

felsorolás
\begin{itemize}
    \item minden függvény-blokk atomikus, vagyis mindig lefut az elejétől a végéig, és nem szakad meg soha
    \item a függvény-blokkban nincsenek elágazások
    \item minden függvény-blokknak van egy maximum nagysága
\end{itemize}


Monospace betűtípus \texttt{fetch}, \texttt{decode} és \texttt{execute}

\paragraph{Nevesített paragrafus.} A {\it dőlt}  idézet~\cite{Ubershaders:ARidiculous}.

Beilleszteni egy képet így kell: A képre mindig kell hivatkozni a szövegből!!!

\begin{figure}[ht]
    \centering
    \includesvg[scale=0.13,inkscapelatex=false]{./kep/utasitas-abra.svg}
	\caption{Utasítás felépítése}
	\label{fig:Utasítás felépítése}
\end{figure}

\newcommand{\CS}{C${}^{\#}$}

\subsection{Így szedjük helyesen a \CS\ nyelvet}

\chapterimage{kep/header2.png} % Chapter heading image
\chapter{Gyakorlati rész}

Még egy kicsit hosszabb, még egy kicsit hosszabb, még egy kicsit hosszabb, még egy kicsit hosszabb, még egy kicsit hosszabb, még egy kicsit hosszabb, még egy kicsit hosszabb, még egy kicsit hosszabb, még egy kicsit hosszabb, még egy kicsit hosszabb, még egy kicsit hosszabb, még egy kicsit hosszabb, még egy kicsit hosszabb, még egy kicsit hosszabb, még egy kicsit hosszabb, még egy kicsit hosszabb, még egy kicsit hosszabb, még egy kicsit hosszabb, még egy kicsit hosszabb, még egy kicsit hosszabb, még egy kicsit hosszabb, még egy kicsit hosszabb, még egy kicsit hosszabb, még egy kicsit hosszabb, még egy kicsit hosszabb, még egy kicsit hosszabb, még egy kicsit hosszabb\footnote{lábjegyzet}, még egy kicsit hosszabb, még egy kicsit hosszabb, még egy kicsit hosszabb, még egy kicsit hosszabb, még egy kicsit hosszabb, még egy kicsit hosszabb, még egy kicsit hosszabb, még egy kicsit hosszabb, még egy kicsit hosszabb, még egy kicsit hosszabb, még egy kicsit hosszabb, még egy kicsit hosszabb, még egy kicsit hosszabb, még egy kicsit hosszabb, még egy kicsit hosszabb, még egy kicsit hossz

\chapterimage{kep/header.png} % Chapter heading image
\chapter*{Befejezés}
\addcontentsline{toc}{chapter}{Befejezés}

Még egy kicsit hosszabb, még egy kicsit hosszabb, még egy kicsit hosszabb, még egy kicsit hosszabb, még egy kicsit hosszabb, még egy kicsit hosszabb, még egy kicsit hosszabb, még egy kicsit hosszabb, még egy kicsit hosszabb, még egy kicsit hosszabb, még egy kicsit hosszabb, még egy kicsit hosszabb, még egy kicsit hosszabb, még egy kicsit hosszabb, még egy kicsit hosszabb, még egy kicsit hosszabb, még egy kicsit hosszabb, még egy kicsit hosszabb, még egy kicsit hosszabb, még egy kicsit hosszabb, még egy kicsit hosszabb, még egy kicsit hosszabb, még egy kicsit hosszabb, még egy kicsit hosszabb, még egy kicsit hosszabb, még egy kicsit hosszabb, még egy kicsit hosszabb\footnote{lábjegyzet}, még egy kicsit hosszabb, még egy kicsit hosszabb, még egy kicsit hosszabb, még egy kicsit hosszabb, még egy kicsit hosszabb, még egy kicsit hosszabb, még egy kicsit hosszabb, még egy kicsit hosszabb, még egy kicsit hosszabb, még egy kicsit hosszabb, még egy kicsit hosszabb, még egy kicsit hosszabb, még egy kicsit hosszabb, még egy kicsit hosszabb, még egy kicsit hosszabb, még egy kicsit hossz


\pagebreak
\chapterimage{kep/header3.png} % Chapter heading image
\chapter*{Resumé}
\addcontentsline{toc}{chapter}{Resumé}
\markboth{}{\sffamily\normalsize{Resumé}}

Autor vyrešil úlohu, ešte trošku dlhšie, ešte trošku dlhšie, ešte trošku dlhšie, ešte trošku dlhšie, ešte trošku dlhšie, ešte trošku dlhšie, ešte trošku dlhšie, ešte trošku dlhšie, ešte trošku dlhšie, ešte trošku dlhšie, ešte trošku dlhšie, ešte trošku dlhšie, ešte trošku dlhšie, ešte trošku dlhšie, ešte trošku dlhšie, ešte trošku dlhšie, ešte trošku dlhšie, ešte trošku dlhšie, ešte trošku dlhšie, ešte trošku dlhšie, ešte trošku dlhšie, ešte trošku dlhšie, ešte trošku dlhšie, ešte trošku dlhšie, ešte trošku dlhšie, ešte trošku dlhšie, ešte trošku dlhšie, ešte trošku dlhšie, ešte trošku dlhšie, ešte trošku dlhšie, ešte trošku dlhšie, ešte trošku dlhšie, ešte trošku dlhšie, ešte trošku dlhšie, ešte trošku dlhšie, ešte trošku dlhšie, ešte trošku dlhšie, ešte trošku dlhšie, ešte trošku dlhšie, ešte trošku dlhšie, ešte trošku dlhšie, ešte trošku dlhšie, ešte trošku dlhšie, ešte trošku dlhšie

% \input{gitlog.tex}

\printbibliography[title=Hivatkozások] % Output book bibliography entries
\addcontentsline{toc}{chapter}{Hivatkozások}
\pagebreak
\thispagestyle{empty}
\mbox{}
\vfill
\begin{Center}
\mbox{\vskip1cm}\EANisbn
\end{Center}

\end{document}
